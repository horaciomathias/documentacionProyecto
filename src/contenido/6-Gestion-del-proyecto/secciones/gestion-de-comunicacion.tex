\section{Gestión de la Comunicación}\label{sec:gestionDeComunicacion}

La comunicación es un aspecto fundamental en cualquier proyecto, ya que permite que los miembros del equipo estén alineados con los objetivos y metas del proyecto,
además de facilitar la resolución de problemas y la toma de decisiones. Por lo tanto, es importante establecer un plan de comunicación que defina los canales,
frecuencia y responsables de la comunicación en el proyecto.

\begin{landscape}
    \begin{table}[H]
        \centering
        \scriptsize
        \begin{tabular}{p{3cm} p{2cm} p{2cm} p{8cm} p{4cm}}
        \hline
        \textbf{Tipo} & \textbf{Canal} & \textbf{Frecuencia} & \textbf{Propósito} & \textbf{Participantes} \\ \hline
        Estado del proyecto y Planificación & Google Meets & Semanal & Visibilizar el estado de las tareas e historias de usuarios. Resolver problemas y dudas, establecer cambios y realizar toma de decisiones. & Equipo de desarrollo \\ \hline
        Tutoría & Microsoft Teams & Semanal & Actualizar el estado del proyecto. & Equipo de proyecto y tutor \\ \hline
        Actualizaciones Informales & WhatsApp & Diario & Comunicaciones rápidas y coordinación & Equipo de proyecto \\ \hline
        Reunión con cliente regular & Google Meets & Cada 9 días & Reunión de descubrimiento y validación de avances regulares. & Equipo de desarrollo, PO y cliente \\ \hline
        Reunión con cliente discovery & Google Meets & Semanal & Reunión para actividades de design thinking. & PO y cliente \\ \hline
        Reunión con cliente Sprint Review & Google Meets & Cada 15 días & Reunión para recolectar feedback sobre los avances realizados en el Sprint. Se realizará el fin de semana posterior a la finalización del Sprint. & Equipo de desarrollo, PO y cliente \\ \hline
        Email posterior a reunión con cliente & Gmail & Email & Establecer una comunicación formal con el cliente para mitigar riesgos sobre el alcance y malentendidos. & Equipo de desarrollo, PO y cliente \\ \hline
        \end{tabular}
        \caption{Matriz de comunicaciones}
        \label{tab:matriDeComunicaciones}
    \end{table}
    
\end{landscape}


