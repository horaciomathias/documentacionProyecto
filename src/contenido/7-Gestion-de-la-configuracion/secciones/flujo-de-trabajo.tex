\section{Elementos de Configuración}\label{sec:elementosDeConfiguracion}

En nuestro equipo, hemos establecido un flujo de trabajo ágil que nos permite gestionar eficientemente las tareas y mantener 
un seguimiento claro del progreso de nuestro proyecto de gestión de incidentes de búsqueda y rescate en el mar.

\subsection{Ciclo de vida de una tarea}
Utilizamos un tablero Kanban para visualizar y gestionar el ciclo de vida de nuestras tareas. Este tablero consta de cinco 
columnas que representan las diferentes etapas por las que pasa cada tarea: To Do, In Progress, Develop, QA y Done.

\subsubsection{To Do}
En esta columna se encuentran todas las tareas pendientes de la iteración actual. Estas tareas han sido previamente definidas 
y priorizadas durante nuestra planificación de sprint. Cada tarea en esta columna está lista para ser tomada por cualquier miembro del equipo.

\subsubsection{In Progress}
Cuando un miembro del equipo comienza a trabajar en una tarea, la mueve a la columna In Progress. Esto indica al resto del equipo que la tarea 
está siendo abordada activamente. Durante esta fase, el desarrollador trabaja en la implementación de la funcionalidad o la resolución del 
problema asociado a la tarea.

\subsubsection{Develop}
Una vez que el desarrollador ha completado su trabajo y ha realizado el merge de su rama a la rama develop (lo que desencadena automáticamente 
el despliegue en nuestro ambiente de desarrollo), la tarea se mueve a la columna Develop. Esta columna representa que la tarea ha sido integrada 
en nuestro ambiente de desarrollo y está lista para ser probada.

\subsubsection{QA}
En esta etapa, realizamos pruebas cruzadas entre los miembros del equipo. Un integrante diferente al que desarrolló la tarea se encarga de probarla 
exhaustivamente en nuestro ambiente de desarrollo. Esto nos permite detectar posibles problemas o mejoras desde una perspectiva fresca. Si se 
encuentran problemas, la tarea puede volver a In Progress para su ajuste.

\subsubsection{Done}
Una vez que una tarea ha pasado satisfactoriamente la fase de QA, se mueve a la columna Done. Esto indica que la tarea está completamente terminada 
y lista para ser incluida en la próxima entrega al cliente. Las tareas en esta columna representan el trabajo completado de la iteración actual.\\
Este flujo de trabajo nos permite mantener un ritmo constante de desarrollo, asegurando que cada tarea pase por las etapas necesarias de implementación 
y validación antes de considerarse completada. Además, facilita la visibilidad del estado del proyecto para todo el equipo y nos ayuda a identificar 
rápidamente cualquier cuello de botella en nuestro proceso de desarrollo.
