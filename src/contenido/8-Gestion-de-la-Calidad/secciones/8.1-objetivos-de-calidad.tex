\section{Objetivos de Calidad}\label{sec:ojetivosDeCalidad}

Los objetivos de calidad están orientados a asegurar que el proyecto no solo se desarrolle de manera efectiva, 
sino que también entregue un producto final que cumpla con las expectativas de los usuarios y las exigencias del 
entorno operacional del MRCC.

\subsection{Objetivos de Proceso}

Estos objetivos buscan mejorar la eficiencia y eficacia en los métodos y procedimientos aplicados durante el desarrollo del proyecto:

\textbf{Adherencia a los plazos.} Es crucial que todas las actividades planificadas para cada fase del proyecto se completen dentro 
del cronograma acordado, incluyendo las entregas iterativas a los interesados clave.

\textbf{Optimización de la velocidad de desarrollo.} Se ha establecido una velocidad esperada de desarrollo de entre 15 y 20 puntos de historia por sprint,
lo que permite una planificación precisa y un flujo constante de trabajo. Esto es vital para asegurar que los entregables se entreguen a tiempo y con la calidad requerida.


\textbf{Reducción del retrabajo.} Mediante la implementación de pruebas unitarias y revisiones de casos de uso, el equipo tiene como objetivo 
limitar el retrabajo a un máximo del 10\% del total de horas trabajadas. Esto se aplica a la corrección de errores y a los cambios solicitados 
por los clientes, asegurando un uso eficiente del tiempo y los recursos.


\subsection{Objetivos de Producto}

Los objetivos del producto se centran en garantizar que el sistema final cumpla con las expectativas de los usuarios del MRCC y esté alineado 
con los estándares de la industria:

\textbf{Aprobación de usuarios.} Se espera obtener una alta satisfacción de los usuarios del MRCC y de los responsables de las operaciones SAR 
en cada fase del proyecto. Se utilizarán encuestas de satisfacción para medir esta aprobación, con una meta de obtener una puntuación entre 4.5 
y 5 en una escala de 5.

\textbf{Resolución de errores.} Un objetivo clave es corregir el 100\% de los errores identificados durante las pruebas beta antes del despliegue
final del sistema. Se dedicará el último sprint y la última versión a la corrección de estos errores, asegurando que el sistema sea robusto y confiable.

\textbf{Conformidad con estándares y mejores prácticas.} El sistema deberá cumplir con los estándares establecidos por la Armada Nacional y con 
las mejores prácticas internacionales en la gestión de incidentes SAR, como las estipuladas en el Manual Internacional IAMSAR. Además, se debe 
garantizar que el sistema sea fácilmente mantenible y escalable para futuras actualizaciones.