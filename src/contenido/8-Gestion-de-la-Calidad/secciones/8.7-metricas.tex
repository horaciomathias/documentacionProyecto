\section{Métricas}\label{sec:metricas}

Las métricas de calidad son esenciales para evaluar y asegurar que el sistema cumpla con los requerimientos de calidad especificados. 
Estas métricas permiten medir tanto la calidad del proceso como del producto final, proporcionando datos clave para la toma de decisiones.

\subsection{Métricas de Proceso}
\subsubsubsection{Tasa de retrabajo}
Tiempo dedicado a corregir defectos después de la fase inicial de desarrollo. La meta es reducir esta tasa mediante la mejora continua 
de los procesos de desarrollo y pruebas.

\subsubsubsection{Velocidad de desarrollo}
Cantidad de puntos de historia completados por sprint. La velocidad esperada es de entre 15 y 20 puntos de historia por sprint, lo que
permite una planificación precisa y un flujo constante de trabajo.

\subsubsection{Métricas de Producto}
\subsubsubsection{Cantidad de defectos por release}
Indicador del número de errores que persisten en cada versión del producto, que se monitorea para asegurar que el producto final sea 
lo más robusto posible.
[cita: https://www.browserstack.com/guide/best-test-efficiency-metrics]

\subsubsubsection{Satisfacción del cliente}
Nivel de satisfacción con el producto y los procesos del proyecto. Se espera alcanzar un nivel de satisfacción superior al 90\%, 
lo que refleja el éxito del proyecto en cumplir con las expectativas del MRCC.
[cita: https://www.netpromoter.com/know/]

\subsubsubsection{Cobertura de Pruebas}
Porcentaje de código cubierto por pruebas automatizadas. Una alta cobertura de pruebas es esencial para minimizar el riesgo de errores 
no detectados en el sistema.