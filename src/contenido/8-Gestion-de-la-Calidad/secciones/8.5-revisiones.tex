\section{Revisiones}\label{sec:revisiones}

Para asegurar un control continuo de la calidad, se implementarán diversas actividades preventivas que permitirán detectar 
cualquier desviación en la calidad de manera temprana. Estas revisiones se centran en diferentes aspectos clave del proyecto 
para garantizar la coherencia y calidad del producto final.

\subsubsubsection{Revisiones de Código}
Las revisiones de código por pares son una de las herramientas más eficaces para mantener la calidad del código fuente. En este proyecto, 
cada cambio significativo en el código será revisado por al menos un miembro del equipo antes de ser integrado en el repositorio principal. 
Este proceso, llevado a cabo a través de herramientas como GitLab Merge Requests, asegura que el código cumpla con los criterios de aceptación 
antes de ser aprobado.\\
Además, para garantizar la consistencia y prevenir errores en las funciones críticas del sistema, como el cálculo del área de búsqueda y el ploteo 
en el mapa, se realizarán inspecciones de código más rigurosas. Estas inspecciones incluirán la participación de expertos en la materia, quienes 
revisarán el código para asegurar que cumpla con los estándares exigidos por la naturaleza crítica de las operaciones de búsqueda y rescate.

\subsubsubsection{Revisiones de Procesos}
Las sprint retrospectives, como parte del marco Scrum adoptado en este proyecto, son fundamentales para la mejora continua del proceso. Estas sesiones 
permiten al equipo reflexionar sobre el trabajo realizado, identificar posibles áreas de mejora y planificar acciones correctivas para futuros sprints. 
Los resultados de estas retrospectivas se registran en un documento específico, disponible en el anexo del proyecto, lo que permitirá un seguimiento 
detallado de las mejoras implementadas y de los problemas que se hayan identificado.

\subsubsubsection{Revisiones de Arquitectura}
Dado que la arquitectura del sistema es un componente crítico para su éxito, se llevará a cabo una revisión exhaustiva de la misma antes de la etapa de 
entrega final. Esta revisión será realizada en conjunto con el Ingeniero Gastón Mousquet, quien aportará su experiencia para validar que la arquitectura 
propuesta no solo cumple con los requisitos técnicos y funcionales, sino que también se alinea con los atributos de calidad esperados, tales como la 
escalabilidad, mantenibilidad y rendimiento. Durante esta revisión, se discutirán los posibles escenarios de uso y se analizarán las tácticas arquitectónicas 
implementadas para asegurar que el sistema pueda soportar las exigencias operacionales del MRCC.

\subsubsubsection{Revisiones de Calidad}
Otra revisión académica fue realizada con Amalia Álvarez, quien mostró un gran interés en las acciones de calidad implementadas por el equipo y en las 
métricas recolectadas. Durante la revisión, Amalia recomendó que el equipo redujera el número de métricas de calidad utilizadas, sugiriendo un enfoque 
más conciso y enfocado en las métricas más críticas.

\subsubsubsection{Revisiones de Documentación}
La documentación del proyecto, al ser un elemento esencial tanto para la fase de desarrollo como para el mantenimiento futuro del sistema, será sometida 
a un proceso de revisión minucioso. Este proceso no solo involucra al tutor del proyecto, sino también a todos los miembros del equipo, quienes revisarán 
y validarán cada sección de la documentación. El proceso de revisión se gestionará mediante un tablero de Trello, donde cada capítulo y sección de la 
documentación será priorizado y asignado al miembro más adecuado del equipo. Esto asegurará que la documentación final sea coherente, precisa y cumpla 
con los estándares académicos y técnicos requeridos.