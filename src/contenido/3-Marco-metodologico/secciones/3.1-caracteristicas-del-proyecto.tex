\section{Características del Proyecto}\label{sec:caracteristicasDelProyecto}

Al tratarse de un proyecto académico la variable tiempo es fija en el contexto de la triple restricción (Alcance, Tiempo , Costo), 
por lo tanto trabajaremos con el enfoque \textit{date driven}. Para elegir el ciclo de vida del proyecto, nos basamos en las siguientes características 
que fueron identificadas:

\subsection{Características del Clientes} 
\textbf{Descripción del cliente.}\\
El Capitán Hugo de Barros, es el Jefe del Centro de Operaciones Tácticas del Comando de la Flota. 
Allí se encuentra el Centro Coordinador de Búsqueda y Rescate en el Mar (MRCC), el cual opera bajo la administración de la Armada Nacional, 
una de las tres fuerzas que componen las Fuerzas Armadas Uruguayas, dirigidas por el Ministerio de Defensa. 
La misión del MRCC comprende la planificación, control, coordinación y ejecución de la totalidad de las operaciones marítimas de Búsqueda y 
Rescate en el área de jurisdicción de la Armada Nacional, a fin de preservar la salvaguardia de la vida humana en el mar, y es su competencia 
la coordinación de los siniestros protagonizados por cualquier tipo de nave o aeronave, en las aguas jurisdiccionales y de interés nacional del 
Río Uruguay, Río de la Plata, Océano Atlántico y aguas interiores de jurisdicción de la Armada.
En esa unidad, la Armada Nacional mantiene una guardia permanente, administrada por el capitán Hugo de Barros. 

\textbf{Cliente disponible.}\\
El cliente tiene muy buena disposición para poder colaborar durante todo el proyecto. Nuestro contacto principal hace un 
servicio de guardia de 24 horas cada 9 días en el puesto de trabajo donde se está realizando el proyecto.

\textbf{Conocimiento del dominio.}\\
El cliente tiene muy buena disposición para poder colaborar durante todo el proyecto. Nuestro contacto principal hace un 
servicio de guardia de 24 horas cada 9 días en el puesto de trabajo donde se está realizando el proyecto.



\subsection{Características del Producto} 

Requerimientos obligatorios que pueden llegar a ser complejos por lo que a continuación se describe:

\textbf{Naturaleza Única de Cada Incidente.}\\
Cada incidente de búsqueda y rescate es único, con su propio conjunto de variables y desafíos, desde las condiciones 
climáticas y corrientes marinas hasta las características específicas de la embarcación en peligro y su tripulación. 
No hay dos incidentes que sean iguales y por lo tanto, la causa y efecto se pueden comprender claramente solo después de que 
se ha respondido al incidente.

\textbf{Ambigüedad y Dinamismo de las Condiciones Marítimas.}\\
Las condiciones marítimas son inherentemente impredecibles y cambiantes. La relación entre las acciones de búsqueda y rescate 
y sus resultados no siempre es predecible, y el éxito de una misión puede depender de factores inesperados que solo se revelan 
a través de la experiencia y el análisis retrospectivo.\\

\textbf{Conocimiento del dominio.}\\
El cliente tiene muy buena disposición para poder colaborar durante todo el proyecto. Nuestro contacto principal hace un 
servicio de guardia de 24 horas cada 9 días en el puesto de trabajo donde se está realizando el proyecto.
Existen una serie de requerimientos flexibles que pueden ser implementados de forma opcional, los cuales en gran medida fueron 
propuestos por el propio equipo dado que 3 de los miembros tienen experiencia de 15 años en la Armada Nacional lo cual 
facilitó tener una visión funcional y técnica al mismo tiempo.


\subsection{Características del equipo} 
\textbf{Tamaño del equipo.}\\
Somos un equipo de 4 miembros de las carreras Ingeniería en Sistemas y Licenciatura en Sistemas, lo cual fortalece las 
habilidades como  grupo. Al ser un equipo reducido es necesario contar con una buena organización y seleccionar herramientas 
de gestión y comunicación para facilitar la coordinación durante todo el proyecto.

\textbf{Experiencia del equipo en la Armada Nacional.}\\
Como se mencionó anteriormente, 3 miembros del equipo trabajaron en la Armada Nacional durante 15 años, lo cual hace que 
conozcamos bastante el dominio. También estos miembros del equipo formaron parte del Servicio de Informática de la Armada, 
lo cual implica que se conoce en detalle la arquitectura y tecnologías requeridas por el cliente.
