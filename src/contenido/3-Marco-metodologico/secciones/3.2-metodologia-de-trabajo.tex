\section{Metodología de Trabajo}\label{sec:metodologiaDeTrabajo}

\subsection{Ciclo de vida} 

Teniendo en cuenta las características del proyecto y las características del equipo se decidió utilizar un ciclo de vida 
incremental.  Dado que el conocimiento en el dominio y en las tecnologías utilizadas por el cliente nos permiten poder obtener 
al principio del proyecto la definición de los requerimientos y la arquitectura.

El proyecto se dividió en dos etapas:

\textbf{Discovery.}\\
Utilizando la metodología \textit{Design Thinking} y el marco de gestión Kanban con el objetivo de relevar los requerimientos a ser ejecutados 
en los primeros \textit{sprint} y también para definir la arquitectura durante un \textit{sprint} 0 en los meses de abril, mayo y junio. 

\textbf{Delivery.}\\
Para cada iteración de duración fija de dos semanas entre los meses de julio a diciembre realizaremos el diseño, codificación y pruebas de forma 
incremental adaptando el marco de gestión \textit{scrum}.

Se utilizará la metodología \textit{Dual Track Scrum} con algunas adaptaciones, porque entendemos que algunos requerimientos podrían requerir de actividades 
de \textit{discovery} durante el proyecto. Se evaluó utilizar solamente \textit{scrum} pero se decidió complementar el inicio generando un primer \textit{backlog} para 
alimentar sus iteraciones con una primera fase de \textit{discovery} a fin de mejorar la Ingeniería de Requerimientos.