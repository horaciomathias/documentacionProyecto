\section{Selección de Tecnologías}\label{sec:seleccionDeTecnologias}

En esta sección se describe el proceso de selección de tecnologías para el desarrollo del Sistema de gestión de incidentes de búsqueda y rescate en el mar (MRCC). 
La selección se basó en diversos criterios relevantes para asegurar que las tecnologías elegidas sean adecuadas para los requisitos del proyecto, ofrezcan un rendimiento óptimo y 
faciliten el desarrollo y mantenimiento del sistema por parte del equipo.

\subsection{Método de ponderación}

Para evaluar las tecnologías disponibles, se utilizó un método de ponderación que asigna puntajes a diferentes aspectos críticos de cada tecnología. Los puntajes se asignan en una escala de 0 a 10, 
distribuidos en las siguientes categorías:

\begin{itemize}
    \item Nula: 0 puntos
    \item Baja: 1-3 puntos
    \item Media: 4-6 puntos
    \item Alta: 7-10 puntos
\end{itemize}

Los aspectos evaluados incluyen:

\begin{itemize}
    \item Madurez de la tecnología: Tiempo y adopción en el mercado.
    \item Documentación disponible: Calidad y cantidad de documentación en línea.
    \item Curva de aprendizaje: Facilidad de aprendizaje y tiempo necesario para adquirir competencia.
    \item Experiencia previa: Nivel de experiencia del equipo con la tecnología, tanto académica como laboral.
\end{itemize}


Para la construcción del backend de nuestra solución web, se consideraron las siguientes tecnologías:

\subsubsection{.Net}
Madurez: Alta. .Net es una plataforma ampliamente adoptada y establecida en el mercado.
Documentación: Alta. Existe una vasta cantidad de documentación y recursos en línea.
Curva de aprendizaje: Media. La plataforma es robusta y puede requerir un tiempo considerable para dominarla completamente.
Experiencia previa: Media. Algunos miembros del equipo tienen experiencia académica y profesional con .Net.

\subsubsection{Next.js}
Madurez: Alta. Next.js ha sido adoptado por una gran cantidad de desarrolladores y empresas.
Documentación: Alta. La documentación oficial y los recursos comunitarios son extensos.
Curva de aprendizaje: Baja. Next.js es conocido por su facilidad de uso y rápida curva de aprendizaje.
Experiencia previa: Alta. Todos los miembros del equipo tienen experiencia previa en JavaScript y frameworks relacionados.

\subsubsection{Comparación de tecnologías}

\begin{table}[H]
    \centering
    \begin{tabular}{c c c c c}
    \hline
    \textbf{Tecnología} & \textbf{Madurez} & \textbf{Documentación} & \textbf{Curva de aprendizaje} & \textbf{Experiencia previa} \\ \hline
    .Net                & Alta            & Alta                   & Media                        & Media                     \\ \hline
    Next.js             & Alta            & Alta                   & Baja                         & Alta                      \\ \hline
    \end{tabular}
    \caption{Comparación de tecnologías}
    \label{tab:comparacionTecnologias}
\end{table}

\subsubsection{Decisión final}
Se decidió utilizar Next.js para el desarrollo del backend debido a su facilidad de uso, amplia documentación y la experiencia 
previa del equipo, lo que reduce el riesgo de implementación y permite un desarrollo más ágil.


\subsection{Base de Datos}

Para la selección del sistema de base de datos, se consideraron las siguientes opciones:

\subsubsection{PostgreSQL}
Madurez: Alta. PostgreSQL es una base de datos relacional muy estable y ampliamente utilizada.
Documentación: Alta. La documentación oficial y los recursos comunitarios son extensos.
Curva de aprendizaje: Media. Requiere conocimientos de SQL y manejo de bases de datos relacionales.
Experiencia previa: Alta. El equipo tiene experiencia previa utilizando PostgreSQL.


\subsubsection{MongoDB}
Madurez: Alta. MongoDB es una base de datos NoSQL muy popular y ampliamente adoptada.
Documentación: Alta. La documentación oficial y los recursos comunitarios son abundantes.
Curva de aprendizaje: Baja. MongoDB es conocida por su flexibilidad y facilidad de uso.
Experiencia previa: Media. Algunos miembros del equipo tienen experiencia previa con MongoDB.


\subsubsection{Comparación de tecnologías}

\begin{table}[H]
    \centering
    \begin{tabular}{c c c c c}
    \hline
    \textbf{Tecnología} & \textbf{Madurez} & \textbf{Documentación} & \textbf{Curva de aprendizaje} & \textbf{Experiencia previa} \\ \hline
    PostgreSQL          & Alta            & Alta                   & Media                        & Alta                      \\ \hline
    MongoDB             & Alta            & Alta                   & Baja                         & Media                     \\ \hline
    \end{tabular}
    \caption{Comparación de tecnologías}
    \label{tab:comparacionTecnologias}
\end{table}

\subsubsubsection{Decisión final}

Se decidió utilizar PostgreSQL para el sistema de base de datos debido a su robustez, la experiencia previa del 
equipo y su capacidad para manejar datos estructurados y transacciones complejas.

\subsection{Memoria Caché}

Para la memoria caché, se consideraron las siguientes opciones:

\subsubsection{Redis}
Madurez: Alta. Redis es ampliamente adoptado y muy estable.
Documentación: Alta. Existe una amplia documentación y soporte comunitario.
Curva de aprendizaje: Baja. Redis es conocido por su simplicidad y eficiencia.
Experiencia previa: Media. Algunos miembros del equipo tienen experiencia previa con Redis.

\subsubsection{Memcached}
Madurez: Alta. Memcached es una tecnología establecida y ampliamente utilizada.
Documentación: Alta. Existe una buena cantidad de documentación y recursos disponibles.
Curva de aprendizaje: Baja. Memcached es sencillo de implementar y utilizar.
Experiencia previa: Nula. Ningún miembro del equipo tiene experiencia con Memcached.


\subsubsection{Comparación de tecnologías}

\begin{table}[H]
    \centering
    \begin{tabular}{c c c c c}
    \hline
    \textbf{Tecnología} & \textbf{Madurez} & \textbf{Documentación} & \textbf{Curva de aprendizaje} & \textbf{Experiencia previa} \\ \hline
    Redis               & Alta            & Alta                   & Baja                         & Media                     \\ \hline
    Memcached           & Alta            & Alta                   & Baja                         & Nula                      \\ \hline
    \end{tabular}
    \caption{Comparación de tecnologías}
    \label{tab:comparacionTecnologias}
\end{table}

\subsubsection{Decisión final}
Se decidió utilizar Redis para la memoria caché debido a su simplicidad, alta performance y la experiencia previa del equipo con 
esta tecnología. Además, Redis ofrece características adicionales como la persistencia de datos, que pueden ser útiles para el 
proyecto MRCC.