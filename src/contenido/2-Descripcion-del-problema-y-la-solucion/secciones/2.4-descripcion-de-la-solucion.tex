\section{Descripción de la Solución}\label{sec:descriptionDeLaSolucion}


Solución tecnológica que gestione los incidentes de búsqueda y rescate en el mar. Esta solución se debe centrar en proporcionar una gestión 
centralizada, integral y eficiente de los incidentes que permita optimizar los recursos, el personal y el tiempo empleado. Para lograrlo debe 
ser imprescindible la capacidad de centralizar y automatizar procesos críticos de la gestión actual de los incidentes tales como: 

\begin{itemize}
    \item Realizar los cálculos de manera rápida y precisa. 
    El tiempo que se ahorra en la realización de los cálculos, considerar factores cambiantes de ciertas variables (como viento y corriente) 
    y no introducir el factor del error humano representa una optimización de los recursos, permite una mejor adaptación de las estrategias, 
    una coordinación eficiente de los involucrados y aumenta significativamente las probabilidades de éxito en los incidentes de rescate;
    \item Gestión de múltiples incidentes en tiempo real. Este es un aspecto fundamental considerando la alta demanda de incidentes que maneja el MRCC;
    \item Visualización detallada en el mapa del plan de rescate y monitoreo de las embarcaciones en tiempo real. 
    \item Acceso a la bitácora inmediata. Mejora la coordinación y reduce los tiempos de comunicación interna y respuesta en la cadena de mando;
    \item Permitir la extracción de estadísticas y reportes. Se presenta como una posibilidad para realizar ajustes continuos y mejoras en las estrategias 
    de búsqueda y rescate teniendo como base los aprendizajes de cada iteración;
    \item Presentar un diseño arquitectónico con bases en la disponibilidad. Asegura la disponibilidad y confiabilidad del sistema, aspecto crítico en operaciones 
    en las que el tiempo es vital.
\end{itemize}