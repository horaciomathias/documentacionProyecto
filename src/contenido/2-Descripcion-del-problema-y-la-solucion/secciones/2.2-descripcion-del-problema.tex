\section{Descripción del Problema}\label{sec:descripcionDelProblema}

Actualmente el MRCC no cuenta con un sistema centralizado de gestión de incidentes, provocando que, cuando se produzca un siniestro, 
se generen ciertas ineficiencias como por ejemplo:

\begin{itemize}
    \item Demoras en las notificaciones a los involucrados;
    \item Demoras en procesamiento de los incidentes;
    \item Dificultad para la toma de decisiones.
    \item Excesiva concentración en el seguimiento del incidente, lo que les impide enfocarse en otras tareas.
    \item Acceso limitado o demorado de información técnica para los involucrados, por ejemplo, los encargados 
    del incidente deben contar con información sobre los procedimientos y la legislación para cada evento particular.
\end{itemize}

Para ilustrar el problema, vamos a examinar un escenario hipotético: un velero que viaja desde Colonia a Piriápolis se 
enfrenta a una tormenta repentina. Durante la tormenta, el miembro más capacitado de la tripulación para navegar sufre un 
accidente y queda inconsciente por lo cual requiere atención médica y a su vez deja al resto de la tripulación sin saber su 
ubicación precisa. El MRCC recibe esta información y se genera un incidente.\\
Antes de enviar un medio SAR al rescate, teniendo en cuenta la última posición conocida y el recorrido que tenía previsto realizar 
el barco del incidente, la guardia del MRCC busca información del tráfico marítimo en un sistema llamado \textit{SEA VISIÓN}, 
e intenta comunicarse con los mismos para ordenarles que asistan al velero.\\
En caso de no poder localizarlo, a partir de la última posición conocida, los datos del viento y de las corrientes se procede hacer un 
cálculo de la posición probable, a través del uso de planillas Excel implicando una demora de alrededor de 20 minutos de un oficial 
entrenado.\\
Luego de hacer el cálculo, dependiendo del escenario del incidente y de los medios disponibles SAR, se realiza un patrón de búsqueda. 
Para el mismo uno de los elementos a tener en cuenta son los incidentes en ubicaciones similares \footnote{los cuales se encuentran archivados 
físicamente en papel, por lo cual se apela a la memoria y experiencia del oficial de guardia}.\\
A continuación, se confecciona un correo electrónico con los datos del incidente y el resultado del patrón de búsqueda y se envía a los medios SAR 
designados. Estos medios SAR designados, a su vez, tiene personal responsable por contactar a toda su tripulación citándolos a presentarse a bordo 
\footnote{dicha citación se realiza tripulante a tripulante, comenzando por quienes se encuentren más alejados geográficamente}. 
Cabe observar que la tripulación de la embarcación son pocos en caso de un buque pequeño, pero cerca de 100 (cien) en caso de un buque de mayor 
porte.\\
Una vez ya desplegados los medios SAR para realizar el rescate, los mismos envían información de su posición a intervalos regulares y también
se solicita información de viento y corriente a efectos de ir actualizando el cálculo con mayor precisión, lo cual implica un retraso  
considerando el tiempo empleado en su elaboración. Posteriormente, esta información es transmitida por radio, aunque también
–a veces– puede ser necesario primero transmitirla a una estación costera para luego ser retransmitida al MRCC, lo cual produce demoras que 
pueden llegar a ser significativas.\\
Una vez localizado el velero, se brinda asistencia según la situación específica. Esta asistencia puede implicar dirigirse hacia el 
embarcadero más cercano a un hospital para darle atención médica a los heridos, remolcar la embarcación si es necesario, solicitar la 
intervención de un helicóptero para evacuar heridos de emergencia mientras el velero es remolcado con el resto de la tripulación, o en 
algunos casos, el médico de la unidad puede encargarse de la atención médica a bordo.\\
Las decisiones sobre cómo proceder se toman considerando varios parámetros como la distancia al embarcadero y al aeropuerto, o la gravedad 
de las heridas, entre otros muchos. Todas las decisiones mencionadas son realizadas y coordinadas desde el MRCC basadas en los 
procedimientos establecidos los cuales contemplan las buenas prácticas del \textit{Manual Internacional IAMSAR}, para lo cual todos los miembros de 
la guardia deben conocer especialmente el CMS (Coordinador de la Misión Sar) quien es responsable de las decisiones que toma durante el 
incidente SAR.\\
Finalmente, una vez completada la operación, se realiza un informe detallado de los eventos ocurridos el cual se archiva físicamente en un 
bibliorato para futuras referencias.\\
Anualmente el MRCC atiende unos 700 incidentes, de los cuales en promedio 50 terminan con el despliegue de un equipo de rescate, de estos 
incidentes se contabilizan alrededor de 10 fallecidos por año.\\
Cabe destacar que para la realización de los procesos de coordinación el MRCC solo cuenta con 6 personas, por lo que la eficiencia en los 
recursos en términos generales es fundamental para el éxito de las operaciones.
