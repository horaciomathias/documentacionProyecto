\section{Principales Interesados}\label{sec:pricipalesInteresados}

El equipo llevó a cabo el estudio de los principales interesados del proyecto. Se identificaron distintos grupos de interés, 
y se evaluaron sus necesidades y expectativas puntuales en relación con la solución propuesta.\\
Se consideró el nivel de influencia que cada uno de estos tenía en el proyecto, de forma de asegurar que sus preocupaciones fueran 
tomadas en cuenta para garantizar su colaboración en la implementación de la solución.\\
Para hacer un estudio detallado de los interesados y evaluar sus necesidades y expectativas se realizó una \textit{matriz de interés e influencia}~\cite{projectmanagement_stakeholder_analysis} 
para categorizarlos en base a los siguientes criterios:


\begin{itemize}
    \item \textbf{Alta influencia -  Alto interés:} Estos son los interesados clave que tienen un gran poder y un alto interés en el 
    éxito del proyecto. Deben ser gestionados de cerca, involucrándose activamente en la toma de decisiones y manteniéndolos informados 
    de manera regular. Su apoyo es crítico para el éxito del proyecto.
    \item \textbf{Alta influencia - Bajo interés:} Aunque no están muy interesados en los detalles diarios del proyecto, su influencia es 
    alta, por lo que es importante mantenerlos satisfechos. Se debe proporcionar información suficiente para asegurarse de que estén 
    contentos con la dirección del proyecto, pero no abrumarnos con demasiados detalles.
    \item \textbf{Baja influencia - Alto interés:} Estos interesados tienen un gran interés en el proyecto pero poca influencia en su éxito 
    o fracaso. Es importante mantenerlos bien informados y asegurarse de que se sientan valorados, aunque no es necesario involucrarlos en 
    las decisiones clave.
    \item \textbf{Baja influencia - Bajo interés:} Estos interesados no tienen mucho poder ni interés en el proyecto. Se debe monitorear su 
    situación para detectar cualquier cambio en su nivel de influencia o interés, pero generalmente no requieren mucha atención ni recursos.
\end{itemize}

El resultado de la clasificación fue el siguiente:

\subsection{Alta Influencia - Alto Interés} 
\textit{Stakeholder Principal: Contralmirante Vizcay (Responsable del MRCC UY)}
\begin{itemize}
    \item \textbf{Características:}
    \begin{itemize}
        \item \textit{Poder de decisión.} Tiene la autoridad final sobre todas las operaciones del MRCC UY, y su respaldo es crucial para la 
        implementación del sistema.
        \item \textit{Visión estratégica.} Está interesado en cómo el sistema mejorará las operaciones generales y la efectividad del MRCC.
        \item \textit{Impacto directo.} Las decisiones del Contralmirante impactan directamente a todos los miembros de la guardia y los recursos 
        SAR disponibles.
    \end{itemize}
    \item \textbf{Preocupaciones:}
    \begin{itemize}
        \item \textit{Eficiencia operativa.} Que el sistema realmente mejore la capacidad de respuesta del MRCC.
        \item \textit{Control y supervisión.} Mantener control sobre cómo se implementa y opera el sistema, asegurando que esté alineado con las 
        metas del MRCC.
        \item \textit{Fiabilidad del sistema.} Garantizar que el sistema sea confiable y no cause interrupciones en las operaciones críticas.
    \end{itemize}
\end{itemize}


\subsection{Alta Influencia / Bajo Interés} 
\textit{Stakeholder: Capitán de Navío Javier Calvo (Jefe del Servicio de Informática de la Armada)}
\begin{itemize}
    \item \textbf{Características:}
    \begin{itemize}
        \item \textit{Poder de decisión técnico.} Responsable de garantizar que el sistema se mantenga operativo desde un punto de vista técnico.
        \item \textit{Involucramiento moderado.} Aunque no está profundamente interesado en los detalles operativos del sistema, su rol es crucial en asegurar su funcionamiento técnico.
    \end{itemize}
    \item \textbf{Preocupaciones:}
    \begin{itemize}
        \item \textit{Mantenimiento del sistema.} Que el sistema sea fácil de mantener y no requiera recursos excesivos de su equipo.
        \item \textit{Integración y soporte.} Asegurar que el sistema se integre bien con las infraestructuras existentes y que su equipo esté preparado para soportarlo.
        \item \textit{Recursos.} Minimizar el impacto en los recursos disponibles de su departamento.
    \end{itemize}
\end{itemize}


\subsection{Media Influencia / Alto Interés} 
\textit{Stakeholder: Guardia del MRCC}
\begin{itemize}
    \item \textbf{Características:}
    \begin{itemize}
        \item \textit{Responsabilidad operativa.} Son los usuarios finales del sistema, responsables de gestionar los incidentes de búsqueda y rescate.
        \item \textit{Uso diario.} Su trabajo se verá directamente afectado por la funcionalidad y eficiencia del nuevo sistema.
    \end{itemize}
    \item \textbf{Preocupaciones:}
    \begin{itemize}
        \item \textit{Facilidad de uso.} Que el sistema sea intuitivo y fácil de usar bajo condiciones de estrés.
        \item \textit{Eficiencia en la gestión de incidentes.} Que el sistema realmente facilite y mejore la gestión de incidentes SAR.
        \item \textit{Capacitación.} Necesidad de formación adecuada para utilizar el sistema de manera efectiva.
    \end{itemize}
\end{itemize}


\subsection{Baja Influencia / Bajo Interés} 
\textit{Stakeholder: Funcionarios del Servicio de Informática de la Armada}
\begin{itemize}
    \item \textbf{Características:}
    \begin{itemize}
        \item \textit{Responsabilidad secundaria.} Tienen un papel de soporte, asegurando la funcionalidad del sistema en caso de fallas en los servidores.
        \item \textit{Involucramiento limitado.} No están profundamente involucrados en el uso diario del sistema ni en sus operaciones.
    \end{itemize}
    \item \textbf{Preocupaciones:}
    \begin{itemize}
        \item \textit{Carga de trabajo adicional.} Preocupación por la responsabilidad adicional que el sistema podría imponerles.
        \item \textit{Minimización de problemas.} Prefieren que el sistema sea lo más estable posible para evitarles problemas adicionales.
        \item \textit{Impacto en la rutina diaria.} Que el sistema no cause interrupciones en sus actividades diarias ni requiera atención constante.
    \end{itemize}
\end{itemize}
