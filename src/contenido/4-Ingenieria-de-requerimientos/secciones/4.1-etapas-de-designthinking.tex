\section{Etapas de \textit{Design Thinking}}\label{sec:etapasDeDesignThinking}

A continuación se mencionan las técnicas de relevamiento utilizadas durante la etapa de discovery, sus objetivos y resultados obtenidos de aplicar cada una de ellas.

\subsection{Empatizar}

El objetivo principal en esta etapa fue comprender profundamente a los usuarios finales del sistema, especialmente a los miembros de la Guardia del MRCC UY. Se puso 
especial énfasis en entender cómo se manejan los incidentes de búsqueda y rescate, y cuáles son los puntos de dolor en los procesos actuales.

\textbf{Hipótesis Inicial del Problema para la Etapa de Empatizar:}\\
"El proceso de cálculo del área probable de búsqueda en el MRCC demora aproximadamente 20 minutos, lo que provoca retrasos significativos en la gestión de incidentes 
de búsqueda y rescate marítimo. Esta demora afecta la capacidad de respuesta rápida del MRCC durante operaciones críticas, comprometiendo la efectividad de las acciones de rescate."


\textbf{Investigación previa}\\
La investigación previa sobre el sistema SAR y los documentos relacionados, como el Manual IAMSAR, el Plan SAR Marítimo Nacional y el Decreto 506/994, permitió obtener un panorama 
completo sobre los marcos legales, operativos y organizativos que rigen las operaciones de búsqueda y rescate en Uruguay. 
Esta revisión destacó la importancia de la coordinación internacional bajo el Convenio SOLAS, la correcta utilización de patrones de búsqueda estandarizados y la planificación eficiente 
de las misiones SAR en cinco etapas clave. Además, se subrayó la necesidad de contar con recursos bien organizados, simulacros anuales para mantener la preparación, y una comunicación efectiva, 
tanto interna como externa, para garantizar el éxito de las operaciones. 
Con este contexto, la planificación de entrevistas puede enfocarse en identificar los puntos críticos de mejora, optimizar procesos y modernizar las herramientas de gestión SAR.

\textbf{Entrevistas}\\
En esta etapa, realizamos entrevistas a usuarios típicos para obtener una aproximación al problema inicial que buscamos solucionar. Al analizar los perfiles de los participantes, identificamos roles 
distintos.
Las entrevistas al Capitán Hugo de Barros y Guillermo Rodríguez, quienes ocupan roles distintos dentro del Maritime Rescue Coordination Center (MRCC), permitieron una comprensión empática y 
multidimensional de los desafíos en la gestión de incidentes de búsqueda y rescate marítimo. 
El objetivo principal de estas entrevistas fue capturar tanto la perspectiva estratégica de un experto como la operativa de un operador con amplia experiencia, asegurando que se abordarán las 
necesidades y dolores específicos de cada nivel dentro del MRCC. 
Al incluir a ambos usuarios, se logró una visión integral que resalta la importancia de soluciones tecnológicas que no solo optimicen la eficiencia y precisión operativa, sino que también 
faciliten la capacitación continua y mejoren la toma de decisiones en tiempo real, respondiendo de manera efectiva a las necesidades reales del personal del MRCC.

\textbf{Customer Journey Map}\\
El propósito de realizar la inmersión en el Maritime Rescue Coordination Center (MRCC) fue obtener una comprensión de las experiencias de los operadores durante la guardia. Al participar y 
observar directamente sus tareas y desafíos cotidianos, buscamos recopilar información detallada para construir un Customer Journey Map que refleja fielmente el proceso y las necesidades de los 
distintos perfiles.\\
Por qué Aplicamos la Técnica:
\begin{itemize}
    \item \textbf{Construir Empatía Auténtica:} Al vivir la experiencia junto a los operadores, pudimos sentir y comprender las presiones, urgencias y responsabilidades que enfrentan, lo que nos permite 
    conectar emocionalmente con sus necesidades y preocupaciones.
    \item \textbf{Identificar Necesidades Latentes:} Muchas veces, los desafíos y obstáculos no son plenamente articulados en entrevistas formales. La observación directa nos permitió identificar 
    problemas subyacentes y oportunidades de mejora que no son evidentes a simple vista.
    \item \textbf{Validar y Profundizar Hallazgos Previos:} Complementar la información obtenida en entrevistas anteriores con observaciones en el entorno real de trabajo, asegurando una comprensión 
    integral y precisa de las situaciones que enfrentan.
    \item \textbf{Contextualizar el Uso de Herramientas y Procesos:} Entender cómo interactúan con los sistemas actuales, cómo manejan la información y cómo se comunican entre ellos, lo que es esencial 
    para diseñar soluciones que se integren eficazmente en su flujo de trabajo.
\end{itemize}

Lo que Esperábamos Obtener:

\begin{itemize}
    \item \textbf{Insights Profundos y Accionables:} Obtener información detallada sobre los puntos de dolor, ineficiencias y desafíos específicos que enfrentan durante la gestión de incidentes, para 
    orientar el diseño de soluciones efectivas.
    \item \textbf{Mejorar la Eficacia de las Soluciones Propuestas:} Asegurar que las recomendaciones y desarrollos futuros estén alineados con las necesidades reales y cotidianas del personal del MRCC, 
    aumentando la probabilidad de adopción y éxito.
    \item \textbf{Fortalecer la Relación con los Usuarios:} Al demostrar un interés genuino por entender su trabajo y desafíos, fomentamos una relación de confianza que facilita la colaboración y 
    receptividad hacia cambios e innovaciones.
    \item \textbf{Desarrollar una Perspectiva Centrada en el Usuario:} La inmersión nos permite mantener al usuario en el centro del proceso de diseño, garantizando que las soluciones no solo sean 
    técnicamente viables sino también relevantes y útiles para quienes las utilizarán.
\end{itemize}



\subsection{Definir}\\
Aclarar y focalizar las necesidades y lograr la comprensión profunda (insights) de lo que hemos detectado durante la etapa de Empatía.  Identificar temas interesantes y patrones recurrentes, comenzar a 
Identificar Perfiles (Personas) para Identificar necesidades e Insights para los temas.

\textbf{Compartir y capturar}\\
Con el objetivo de consolidar y organizar la información obtenida durante la fase de Empatía, realizamos la actividad de “Compartir y Capturar”.
La actividad de "Compartir y Capturar" es esencial para transformar los hallazgos de la fase de empatía en información accionable. Al compartir nuestras observaciones y capturar los aspectos más relevantes 
sobre los usuarios, estamos construyendo una base sólida para definir con precisión el problema que queremos resolver. Esto nos permitirá diseñar soluciones efectivas y centradas en las personas, alineadas 
con las necesidades y desafíos identificados en nuestra investigación.

\textbf{Saturar y agrupar}
Con el objetivo de Identificar temas interesantes y patrones recurrentes, realizamos la actividad de “Saturar y agrupar”. Para visualizar de manera efectiva los problemas y oportunidades identificados, es 
esencial agrupar físicamente los post-its.\\
Este proceso facilita la identificación de temas recurrentes y patrones comunes, lo que a su vez ayuda a enfocar los esfuerzos en las áreas más críticas. La saturación garantiza que se ha capturado toda la 
información relevante, mientras que la agrupación facilita la identificación de temas y patrones esenciales.\\
Este enfoque estructurado no solo mejora la comprensión de los desafíos existentes, sino que también sienta las bases para el desarrollo de soluciones efectivas y colaborativas.


\textbf{Identificar Perfiles}
Comenzar a identificar perfiles es un paso crucial en el proceso de definir el problema porque nos permite entender profundamente quiénes son los usuarios, qué necesidades tienen y cómo interactúan con el 
contexto en el que se desenvuelven.

Esta técnica nos permitió obtener una versión inicial de necesidades e insights, luego el equipo fue re-escribiendo los mismos y se obtuvieron insights refinados teniendo en cuenta el checklist 
[CREATE INSIGHTS STATEMENTS METHOD CHECKLIST] link https://chrome.google.com/webstore/detail/pdf-viewer/oemmndcbldboiebfnladdacbdfmadadm?hl=pt-BR.


\subsection{Idear}\\


\subsection{Prototipar y Probar}\\
