\section{Requerimientos no funcionales}\label{sec:requerimientosNoFuncionales}

A partir de los requerimientos no funcionales, se definieron atributos de calidad que a se explicarán en el capítulo de Arquitectura: 

\subsection{Availability}

\textbf{RNF-01}\\
El sistema deberá garantizar alta disponibilidad y resiliencia operativa, pudiendo funcionar 24/7 sin interrupciones (deben poder seguir gestionando el incidente en un ambiente local, 
sin conexión a internet), incluso durante cortes de energía o de red. Debe incluir mecanismos de failover automático y redundancia de datos.


\subsection{Performance}

\textbf{RNF-02}\\
El sistema debe ofrecer tiempos de respuesta rápidos y manejo eficiente de los recursos, especialmente en situaciones críticas y de alta carga operativa.


\subsection{Security}

\textbf{RNF-03}\\
El sistema debe garantizar la privacidad de los datos en flujo. Esto incluye el cifrado de datos en tránsito para prevenir la interceptación y el acceso no autorizado.

\textbf{RNF-04}\\
Los usuarios deberán ser identificados, autenticados y autorizados cada vez que interactúen con la plataforma.

\textbf{RNF-05}\\
El sistema debe mantener la mayor superficie posible de la aplicación de manera privada.

\textbf{RNF-06}\\
Se requerirá uso de roles con distintos privilegios, para controlar el acceso a funcionalidades e información que el sistema maneja. Los roles del sistema son:
\begin{itemize}
    \item Administrador, acceso completo al sistema.
    \item Operador, para gestionar los incidentes.
    \item Jefe, para monitorear los incidentes
\end{itemize}


\subsection{Usability}

\textbf{RNF-07}\\
El sistema debe ser intuitivo y fácil de aprender, permitiendo a los usuarios entender rápidamente si el software es adecuado para sus necesidades.


\subsection{Modifiability}

\textbf{RNF-08}\\
Los cambios de configuración deben poder realizarse sin causar interrupciones significativas en el servicio. Esto incluye la capacidad de actualizar, ajustar y 
mejorar el sistema mientras se mantiene su disponibilidad y funcionalidad operativa.

\textbf{RNF-09}\\
El código y la arquitectura del sistema deben facilitar el mantenimiento y las futuras actualizaciones. Incluir documentación detallada y código fuente comentado 
para apoyar operaciones continuas y desarrollo independiente por parte del equipo de TI de la Armada.


\subsection{Deployability}

\textbf{RNF-10}\\
El sistema debe ser adaptable a diferentes entornos de hardware y software de manera efectiva, asegurando su operación continua y eficiente independientemente de las plataformas subyacentes.

\subsection{Scalability}

\textbf{RNF-11}\\
Capacidad de escalar de manera eficiente para manejar incrementos en la carga de trabajo o en el número de incidentes gestionados simultáneamente sin degradar el rendimiento.

\subsection{Interoperability}

\textbf{RNF-12}\\
El sistema debe poder integrarse con APIs externas para obtener datos meteorológicos y otros datos relevantes en tiempo real.